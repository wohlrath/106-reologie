\section*{Závěr}
Pyknometrickou metodou jsme změřili hustotu glycerinu a podle ní určili jeho koncentraci přibližně \SI{97.5}{\percent}.
Rotačním viskozimetrem jsme ověřili, že je glycerin newtonovská kapalina a změřili jsme při teplotě \SI{25.6}{\degreeCelsius} jeho dynamickou viskozitu \SI{720(30)}{\milli\pascal\s}.

Rotačním viskozimetrem jsme také změřili závislost zdánlivé viskozity škrobu na rychlosti deformace.
S rostoucí rychlostí deformace viskozita klesá, a to mocninně s exponentem přibližně \num{-0.58}.
Ověřili jsme, že škrob je nenewtonovská kapalina.

Kuličkovým viskozimetrem jsme změřili závislost viskozity glycerinu (koncentrace \SI{99.5}{\percent}) na teplotě v rozmezí \SI{25}{\degreeCelsius} až \SI{35}{\degreeCelsius}.
Viskozita s rostoucí termodynamickou teplotou exponenciálně klesá.
Ze závislosti jsme určili aktivační energii $\epsilon_A = $(\num{1.00(2)})\,\SI{e10}{\joule} 

Všechny uvedené odchylky jsou standardní ($P=\SI{68}{\percent}$).