\section*{Výsledky měření}
Teplota v místnosti byla \SI{25.6}{\degreeCelsius}.
Atmosférický tlak byl \SI{992}{\hecto\pascal}.
Relativní vlhkost vzduchu byla \SI{33}{\percent}.
Hustota vzduchu $\sigma$ je při těchto podmínkách přibližně \SI{1.18(1)}{\kg\per\m\cubed}.


V rotačním viskozimetru jsme měřili glycerin a škrob.
Teplota obou kapalin byla \SI{25.6}{\degreeCelsius}.
V kuličkovém pouze glycerin při teplotách v rozmezí \SI{25}{\degreeCelsius} až \SI{35}{\degreeCelsius}.

Naměřené hodnoty viskozity glycerinu jsou uvedeny v tabulce \ref{tab::glyc} zaneseny do grafu \ref{grp::rotglyc}.
Viskozita glycerinu na frekvenci otáčení nijak zřejmě nezávisí (Pearsonův korelační koeficient \num{0.04}).
Glycerin proto považujeme za newtonovskou kapalinu.
Viskozitu určíme metodou nejmenších čtverců, $\eta = $\SI{720(30)}{\milli\Pa\s}.

\begin{tabulka}[htbp]
\centering
\begin{tabular}{ccc|ccc}
rotor & frekvence (\si{\per\minute}) & viskozita (\si{\milli\pascal\s}) & rotor & frekvence (\si{\per\minute}) & viskozita (\si{\milli\pascal\s}) \\
\hline 
\multirow{9}{*}{L1} & \num{0.6} & 861 & \multirow{6}{*}{L2} & 5 & 828 \\
& 1 & 801 & & 6 & 792 \\
& 1.5 & 754 & & 10 & 644 \\
& 2 & 721 & & 12 & 605 \\
& 2.5 & 657 & & 20 & 610 \\
& 3 & 653 & & 30 & 590 \\
& 4 & 667 & & &  \\
& 5 & 661 & \multirow{5}{*}{L3} & 20 & 940 \\
& 6 & 656 & & 30 & 760 \\
&   &     & & 50 & 760 \\
\multirow{2}{*}{L4} & 100 & 740 & & 60 & 740 \\
& 200 & 720 & & 100 & 750 \\
\end{tabular}
\caption{Naměřená viskozita glycerinu rotačním viskozimetrem}
\label{tab::glyc}
\end{tabulka}


\begin{graph}[htbp] 
\centering
\input{glycerin.tex}
\caption{Dynamická viskozita glycerinu měřená rotačním viskozimetrem s různými rotory (logaritmické měřítko na vodorovné ose)}
\label{grp::rotglyc}
\end{graph}

U škroby je naopak závislost na rychlosti otáčení zřejmá, škrob je tedy nenewtonovská kapalina.
Naměřené hodnoty jsou uvedeny v tabulce \ref{tab::skrob} a zaneseny do grafu \ref{grp::rotskrob}.
Závislost fitujeme funkcí $\eta(F)=A \cdot F^{n-1}$, kde $F$ je frekvence otáčení rotoru v otáčkách za minutu.
Po dosazení $D = 2 \cdot 2\pi \cdot F/60$ dostáváme konstanty v \eqref{eq::fitnenewton}
\begin{equation}
m=\SI{4900(400)}{\milli\Pa\s\tothe{n}} \qquad n = \num{0.42(3)}
\end{equation}

Zdánlivá viskozita se zvyšující se rychlostí klesá $(n < 1)$, škrob je tedy pseudoplastická látka \cite{skripta}.


\begin{tabulka}[htbp]
\centering
\begin{tabular}{ccc|ccc}
rotor & frekvence (\si{\per\minute}) & viskozita (\si{\milli\pascal\s}) & rotor & frekvence (\si{\per\minute}) & viskozita (\si{\milli\pascal\s}) \\
\hline 
\multirow{7}{*}{L2} & 10 & 3000 & \multirow{2}{*}{L2} & 5 & 4888 \\
& 12 & 2670 & & 6 & 4405 \\
& 20 & 1980 & & & \\
& 30 & 1910 & & & \\
& 50 & 1310 & & & \\
& 60 & 1200 & & & \\
& 100 & 950 & & & \\
\end{tabular}
\caption{Naměřené zdánlivé viskozity škrobu rotačním viskozimetrem}
\label{tab::skrob}
\end{tabulka}


\begin{graph}[htbp] 
\centering
\input{skrob.tex}
\caption{Zdánlivá viskozita škrobu v závislosti na rychlosti otáčení (logaritmické měřítko u obou os)}
\label{grp::rotskrob}
\end{graph}

Kuličkový viskozimetr byl naplněn bezvodým glycerinem (\SI{99.5}{\percent}), jeho hustotu udává \cite{skripta} při naší teplotě \SI{1.257(1)}{\g\per\cm\cubed}.
Naměřené hodnoty jsou uvedeny v tabulce \ref{tab::teplota} a zaneseny do grafů \ref{grp::kulickovej} a \ref{grp::kulickovejadapt}.
Ze směrnice přímky \eqref{eq::primka} určíme aktivační energii $\epsilon_A = $(\num{1.00(2)})\,\SI{e10}{\joule}.

\begin{tabulka}[htbp]
\centering
\begin{tabular}{ccc}
teplota (\si{\degreeCelsius}) & čas (\si{\s}) & viskozita (\si{\milli\pascal\s}) \\ \hline 
25.2	&		183	& 888 \\
26.2	&		173	& 839 \\
27.1	&		160	& 776 \\
28.2	&		146	& 708 \\
29.1	&		135	& 655 \\
30.1	&		124	& 602 \\
31.1	&		115	& 558 \\
32.1	&		107	& 519 \\
33		&		99	& 480 \\
34		&		92	& 446 \\
35		&		86	& 417 \\
\end{tabular}
\caption{Naměřené hodnoty kuličkovým viskozimetrem}
\label{tab::teplota}
\end{tabulka}


\begin{graph}[htbp] 
\centering
\input{kulickovej.tex}
\caption{Závislost viskozity glycerinu na teplotě, měřeno kuličkovým viskozimetrem}
\label{grp::kulickovej}
\end{graph}

\begin{graph}[htbp] 
\centering
\input{kulickovejadapt.tex}
\caption{Závislost viskozity glycerinu na teplotě (přizpůsobené osy)}
\label{grp::kulickovejadapt}
\end{graph}

Pyknometrickou metodou jsme změřili hustotu glycerinu, který jsme používali v rotačním viskozimetru.
Hmotnost prázdného pyknometru $m_1 = $\SI{24.805(5)}{\g}.
Hmotnost pyknometru naplněného destilovanou vodou $m_2 =$\SI{49.678(5)}{\g}.
Hmotnost pyknometru naplněného použitým glycerinem $m_3 = $\SI{56.042(5)}{\g}.
Hustota destilované vody při \SI{25.6}{\degreeCelsius} je \SI{997.0(5)}{\g\per\cm\cubed} \cite{hustotavody}.
Hustotu použitého glycerinu jsme podle \eqref{eq::pyknometr} určili jako \SI{1251.8(6)}{\g\per\cm\cubed}.
Podle \cite{skripta} má glycerin tuto hustotu při koncentraci přibližně \SI{97.5}{\percent}.