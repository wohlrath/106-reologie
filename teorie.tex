\section*{Teoretická část}
Dynamickou vikozitu kapaliny $\eta$ můžeme definovat vztahem \cite{skripta}
\begin{equation}
\tau = \eta \cdot D \,,
\end{equation}
kde $\tau$ je smykové napětí a $D$ je rychlost smykové deformace.

Pokud je $\eta$ konstanta a nezívisí na rychlosti deformace, nazýváme kapalinu newtonovskou.
Pokud $\eta$ na rychlosti závisí, mluvíme o zdánlivé viskozitě a kapalinu nazýváme nenewtonovskou.
Pro některé nenewtonovské kapaliny platí mocninný zákon \cite{skripta}
\begin{equation} \label{eq::fitnenewton}
\eta = m \cdot D^{n-1} \,,
\end{equation}
kde $m$ je konstanta a $n$ je číslo.

Změnu viskozity s teplotou můžeme charakterizovat vztahem \cite{skripta}
\begin{equation}
\eta(T) = C \cdot \exp(\frac{\epsilon_A}{kT}) \,,
\end{equation}
kde $C$ je konstanta, $\epsilon_A$ je aktivační energie, $k$ je Boltzmannova konstanta a $T$ je termodynamická teplota.
Po zlogaritmování dostaneme
\begin{equation} \label{eq::primka}
\ln(\eta) = \ln(C) + \frac{\epsilon_A}{k} \cdot \frac{1}{T} \,,
\end{equation}
tedy rovnici přímky v proměnných $\ln{\eta}$ a $1/T$.

Viskozitu budeme měřit rotačním a kuličkovým viskozimetrem značky HAAKE.

Rotační viskozimetr je vybaven čtyřmi rotory různé velikosti ozančenými \emph{L1}, \emph{L2}, \emph{L3} a \emph{L4} a umožňuje volit frekvenci otáčení.

V kuličkovém viskozimetru měříme čas $t$, za který kulička ve viskózní kapalině urazí vzdálenost \SI{100}{\mm}.
Kuličkový viskozimetr je připojen k přístroji, který umožňuje nastavit teplotu kapaliny.
Viskozitu určíme ze změřeného času $t$ pomocí vztahu \cite{skripta}
\begin{equation} \label{eq::teplotavypocet}
\eta = K \cdot (\rho_1 - \rho_2) \cdot t \,,
\end{equation}
kde $\rho_1$ je hustota kuličky, $\rho_2$ hustota kapaliny a $K$ je konstanta kuličky.
Byla použita ocelová kulička o hmotnosti \SI{14.92}{\g}, průměru \SI{15.19}{mm} a hustotě $\rho_1 = $\SI{8.127}{\g\per\cm\cubed}.
Pro takovou kuličku uvádí \cite{skripta} konstantu kuličky \mbox{$K = \SI{0.7061}{\milli\Pa\per\cm\cubed\per\g}$}.

Viskozita glycerinu je silně závislá na složení a v rotačním viskozimetru neměříme čistý glycerin.
Jeho koncentraci změříme pyknometrickou metodou \cite{pyknometr}.
Změříme hmotnost prázdného pyknometru $m_1$, hmotnost pyknometru naplněného vodou (o známé hustotě) $m_2$ a hmotnost pyknometru naplněného glycerinem $m_3$.
Hustotu glycerinu určíme podle vztahu \cite{skripta}
\begin{equation} \label{eq::pyknometr}
\rho_{glycerin} = \frac{m_3 - m_1}{m_2 - m_1} \cdot (\rho_{voda} - \sigma) + \sigma \,,
\end{equation}
kde $\sigma$ je hustota vzduchu.